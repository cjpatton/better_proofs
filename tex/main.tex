\documentclass{article}
\usepackage{amsmath, amsthm}
\usepackage{listings}
\usepackage{listings-rust}
\usepackage[margin=1in]{geometry}
\usepackage{sourcecodepro}
\usepackage[T1]{fontenc}
\usepackage{hyperref}
\usepackage[svgnames]{xcolor}

\lstset{
  language=rust,
  style=boxed,
}

% Colors

\definecolor{darkgreen}{RGB}{30,97,0}

% Theorems

\newtheorem{claim}{Claim}
\newtheorem{definition}{Definition}
\newtheorem{theorem}{Theorem}
\newtheorem{remark}{Remark}

% Macros

\newcommand{\authnote}[3]{%
  {
    \color{#1}%
    \textbf{[#2:~}%
    {%
      #3%
    }%
    \textbf{]}%
  }%
}
\newcommand{\cp}[1]{\authnote{darkgreen}{Chris P.}{#1}}

\newcommand{\Adv}[1]{\textrm{\textup{Adv}}^{\textup{#1}}}
\newcommand{\Prob}[1]{\Pr\hspace{-1pt}\big[\,#1\,\big]}
\newcommand{\code}[1]{\textrm{\textup{\lstinline{#1}}}}

\title{Better proofs}
\author{Christopher Patton (chrispatton+ietf@gmail.com)}

\begin{document}

\maketitle

\begin{abstract}
  \cp{TODO.}
\end{abstract}

\section{Introduction}

\cp{TODO.}

\section{Notation}

All algorithms, including adversaries, are rust programs without $\code{async}$
semantics.
%
Syntax and oracles are defined as rust traits.
%
Let $\code{block}$ be a rust code block whose last statement is an expression
that evaluates to a $\code{bool}$.
%
We denote by $\Prob{\code{block}}$ the probability that execution of
$\code{block}$ results in $\code{true}$.
%
By convention, an adversary's runtime includes the time it takes to initialize
its game and evaluate its oracle queries.

\section{Example: Hybrid Encryption}

In this section we consider the hybrid public-key encryption (PKE) scheme of
\cite[Chapter~15]{joy}.
%
The proof demonstrates two types of game transitions: \emph{equivalent-rewrite}
and \emph{equivalent-by-reduction}.

\subsection{Syntax}

A PKE scheme implements the $\code{PubEnc}$ trait listed in
Figure~\ref{fig/pubenc/syntax}.
%
Let $\code{enc: E}$ implement $\code{PubEnc}$.
%
We say $\code{enc}$ is \emph{correct} if for all $\code{m} \in
\code{E::Plaintext}$ it holds that
%
\[
  \Prob{
    \code{let (pk, sk) = enc.key_gen();
    enc.decrypt(sk, enc.encrypt(pk, m)) == Some(m)}
  } = 1 \,.
\]

% ../src/joy/hybrid/mod.rs
\begin{figure}
\begin{lstlisting}
pub trait PubEnc {
    type PublicKey;
    type SecretKey;
    type Plaintext;
    type Ciphertext;
    fn key_gen(&self) -> (Self::PublicKey, Self::SecretKey);
    fn encrypt(&self, pk: &Self::PublicKey, m: &Self::Plaintext) -> Self::Ciphertext;
    fn decrypt(
        &self,
        sk: &Self::SecretKey,
        c: &Self::Ciphertext,
    ) -> Option<Self::Plaintext>;
}
\end{lstlisting}
  \caption{Syntax of public-key encryption (PKE).}
  \label{fig/pubenc/syntax}
\end{figure}

\subsection{Security}

Security under chosen plaintext attack (CPA) is defined by the $\code{PubCpa}$
game in Figure~\ref{fig/pubenc/security}.
%
It captures a left-or-right notion of indistinguishability by giving the
attacker an oracle \hyperref[sec/traits]{$\code{LeftOrRight}$} that encrypts
one of two messages chosen by the attacker and returns the ciphertext.
%
Which message is encrypted is controlled by a boolean $\code{right}$ used to
initialize the game.

\begin{definition}[{\cite[Definition 15.1]{joy}}]
  Let~$A$ implement \hyperref[sec/traits]{$\code{Distinguisher}$}.
  %
  Let~$\code{enc}$ implement \hyperref[fig/pubenc/syntax]{$\code{PubEnc}$}.
  %
  Define the advantage of attacker~$A$ in breaking the security of
  $\code{enc}$ under CPA as
  %
  \[
    \Adv{cpa}_{\code{enc}}(A) =
      \Prob{A\code{.play(PubCpa::init(enc, false))}} -
      \Prob{A\code{.play(PubCpa::init(enc, true))}} \,,
  \]
  %
  where $\code{PubCpa}$ is as defined in Figure~\ref{fig/pubenc/security}.
  %
  Informally, we say $\code{enc}$ is secure under CPA if every efficient
  attacker has small advantage.
\end{definition}

\begin{remark}
  This definition implicitly requires length hiding, since there is no
  requirement that the challenge plaintext $\code{m_left}$ and $\code{m_right}$
  have the same length. For most PKE schemes, the length of the ciphertext is a
  function of the length of the plaintext.
\end{remark}

% ../src/joy/hybrid/mod.rs
\begin{figure}
\begin{lstlisting}
pub struct PubCpa<E: PubEnc> {
    enc: E,
    pk: E::PublicKey,
    right: bool,
}
impl<E: PubEnc> PubCpa<E> {
    pub fn init(enc: E, right: bool) -> Self {
        let (pk, _sk) = e.key_gen();
        Self { enc, pk, right }
    }
}
impl<E: PubEnc> GetPublicKey for PubCpa<E> {
    type PublicKey = E::PublicKey;
    fn get_pk(&self) -> &E::PublicKey {
        &self.pk
    }
}
impl<E: PubEnc> LeftOrRight for PubCpa<E> {
    type Plaintext = E::Plaintext;
    type Ciphertext = E::Ciphertext;
    fn left_or_right(
        &self,
        m_left: &E::Plaintext,
        m_right: &E::Plaintext,
    ) -> E::Ciphertext {
        if self.right {
            self.enc.encrypt(&self.pk, m_right)
        } else {
            self.enc.encrypt(&self.pk, m_left)
        }
    }
}
\end{lstlisting}
  \caption{Game \lstinline{PubCpa} for defining security of PKE under chosen
  plaintext attack (CPA).}
  \label{fig/pubenc/security}
\end{figure}


\subsection{Construction}

In practice, PKE schemes are constructed by combining symmetric encryption with
a mechanism for encapsulating the encryption key to the intended
recipient~\cite{rfc9180} (that is, the holder of the secret key corresponding
to the public encapsulation key).
%
By way of gentle introduction to this paradigm, Mike Rosulek gives a
construction of CPA-secure, hybrid encryption in ``The Joy of
Cryptography''~\cite{joy}.
%
We recall this construction in Figure~\ref{fig/hybrid}.

\begin{figure}[t]
\begin{lstlisting}
pub struct Hybrid<P, S> {
    pub pub_enc: P,
    pub sym_enc: S,
}
impl<P, S> PubEnc for Hybrid<P, S>
where
    S: SymEnc,
    P: PubEnc<Plaintext = S::Key>,
    Standard: Distribution<S::Key>,
{
    type PublicKey = P::PublicKey;
    type SecretKey = P::SecretKey;
    type Plaintext = S::Plaintext;
    type Ciphertext = (P::Ciphertext, S::Ciphertext);
    fn key_gen(&self) -> (P::PublicKey, P::SecretKey) {
        self.pub_enc.key_gen()
    }
    fn encrypt(
        &self,
        pk: &P::PublicKey,
        m: &S::Plaintext,
    ) -> (P::Ciphertext, S::Ciphertext) {
        let tk = thread_rng().gen(); // "temporary key"
        let c_pub = self.pub_enc.encrypt(pk, &tk);
        let c_sym = self.sym_enc.encrypt(&tk, m);
        (c_pub, c_sym)
    }
    fn decrypt(
        &self,
        sk: &Self::SecretKey,
        (c_pub, c_sym): &(P::Ciphertext, S::Ciphertext),
    ) -> Option<Self::Plaintext> {
        let tk = self.pub_enc.decrypt(sk, c_pub)?;
        let m = self.sym_enc.decrypt(&tk, c_sym)?;
        Some(m)
    }
}
\end{lstlisting}
  \caption{Hybrid PKE scheme of \cite[Construction 15.8]{joy}.}
  \label{fig/hybrid}
\end{figure}

This construction transforms a symmetric encryption scheme $\code{sym_enc: S}$
into a PKE scheme.
%
The syntax and security of symmetric encryption are defined similarly to PKE,
except there is no public key; we defer the formal definition to
Appendix~\ref{sec/symenc}.
%
The transformation involves a PKE scheme $\code{pub_enc: E}$ with a small
plaintext space. In particular, we require that plaintext space of
$\code{pub_enc}$ is the same as the key space for $\code{sym_enc}$. Formally,
$\code{E: PubEnc<Plaintext = S::Key>}$ (see line 8).

To encrypt a plaintext, we choose a ``temporary key'' $\code{tk}$ uniform
randomly from the key space of $\code{sym_enc}$.
%
We then encrypt $\code{tk}$ using the PKE scheme $\code{pub_enc}$.
%
Finally, we encrypt the plaintext under $\code{tk}$ and output both
ciphertexts.

\begin{theorem}\label{thm/hybrid}
  Suppose that $\code{pub_enc}$ and $\code{sym_enc}$ satisfy the constraints of
  Figure~\ref{fig/hybrid} and let $\code{hybrid} = \code{Hybrid\{
    pub_enc, sym_enc \}}$.
  %
  Then for every $\code{hybrid}$-attacker $A$ there exist a
  $\code{sym_enc}$-attacker $B$ and a $\code{pub_enc}$-attacker $C$ with
  the same runtime as $A$ for which
  %
  \[
    \Adv{cpa}_\code{hybrid}(A) \leq
      \Adv{cpa}_\code{sym_enc}(B) +
      2\cdot\Adv{cpa}_\code{pub_enc}(C) \,.
  \]
\end{theorem}

\begin{proof}
  The proof is by a game-playing argument.
%
We begin with game $\code{G1}$ listed below:

\begin{lstlisting}
pub struct G1<P: PubEnc, S> {
    enc: Hybrid<P, S>,
    pk: P::PublicKey,
}
impl<P: PubEnc, S> G1<P, S> {
    pub fn init(enc: Hybrid<P, S>) -> Self {
        let (pk, _sk) = enc.pub_enc.key_gen();
        Self { enc, pk }
    }
}
impl<P: PubEnc, S> GetPublicKey for G1<P, S> {
    type PublicKey = P::PublicKey;
    fn get_pk(&self) -> &P::PublicKey {
        &self.pk
    }
}
impl<P, S> LeftOrRight for G1<P, S>
where
    S: SymEnc,
    P: PubEnc<Plaintext = S::Key>,
    Standard: Distribution<S::Key>,
{
    type Plaintext = S::Plaintext;
    type Ciphertext = (P::Ciphertext, S::Ciphertext);
    fn left_or_right(
        &self,
        m_left: &S::Plaintext,
        _m_right: &S::Plaintext,
    ) -> (P::Ciphertext, S::Ciphertext) {
        let mut rng = thread_rng();
        let tk_left = rng.gen();
        let _tk_right = rng.gen(); // new temporary key (unused)
        let c_pub = self.enc.pub_enc.encrypt(&self.pk, &tk_left);
        let c_sym = self.enc.sym_enc.encrypt(&tk_left, m_left);
        (c_pub, c_sym)
    }
}
\end{lstlisting}
%
This game was obtained from \hyperref[fig/pubenc/security]{PubEnc} by fixing
$\code{right == false}$, unrolling the call to $\code{self.enc.encrypt()}$ in
the $\code{LeftOrRight}$ oracle, and adding code for generating a new temporary
key, $\code{_tk_right}$. Let us call this new key (unused for the moment) the
``right key'' and the existing key the ``left key''.
%
\begin{claim}[Equivalent-rewrite]
  For all $A$,
  \[
     \Prob{A\code{.play(PubCpa::init(hybrid, false))}} =
     \Prob{A\code{.play(G1::init(hybrid))}} \,.
  \]
\end{claim}
%
%\cp{The main goal for this document is to identify the best methodology for
%establishing claims like this. The goal here seems simple, at least in
%principle: prove that the programs in this equality are equivalent, and thus
%the output being $\code{true}$ is equiprobable in both experiments.}

Next, game $\code{G2}$ is obtained by applying the following patch to the
$\code{LeftOrRight}$ oracle. By this we mean removing the lines beginning with
``$\code{-}$'' and and adding the lines beginning with ``$\code{+}$'':
%
\begin{lstlisting}[style=patch]
    fn left_or_right(
        &self,
        m_left: &S::Plaintext,
-       _m_right: &S::Plaintext,
+       m_right: &S::Plaintext,
    ) -> (P::Ciphertext, S::Ciphertext) {
         let mut rng = thread_rng();
         let tk_left = rng.gen();
-        let _tk_right = rng.gen(); // new temporary key (unused)
-        let c_pub = self.enc.pub_enc.encrypt(&self.pk, &tk_left);
+        let tk_right = rng.gen(); // new temporary key
+        let c_pub = self.enc.pub_enc.encrypt(&self.pk, &tk_right);
         let c_sym = self.enc.sym_enc.encrypt(&tk_left, m_left);
         (c_pub, c_sym)
     }
\end{lstlisting}
%
The patched oracle encrypts the right key rather than the left. The left key is
still used to produce $\code{s_sym}$, so $\code{c_sym}$ is now independent of
$\code{c_pub}$.

We can reduce the attacker's ability to distinguish between $\code{G1}$ and
$\code{G2}$ to the CPA security of $\code{pub_cpa}$.
%
Consider the following reduction, which transforms a $\code{hybrid}$ attacker
into a $\code{pub_enc}$ attacker. For the moment, fix $\code{sim_right ==
false}$; we will use $\code{sim_right == true}$ in a later step.
%
\begin{lstlisting}
pub struct FromHybridToPubEnc<P, S>
where
    P: PubEnc,
{
    sym_enc: S,
    game: PubCpa<P>, // Instance of the game for `pub_enc`.
    sim_right: bool,
}
impl<P, S> FromHybridToPubEnc<P, S>
where
    P: PubEnc,
{
    pub fn init(game: PubCpa<P>, sym_enc: S, sim_right: bool) -> Self {
        Self {
            game,
            sym_enc,
            sim_right,
        }
    }
}
impl<P, S> GetPublicKey for FromHybridToPubEnc<P, S>
where
    P: PubEnc,
{
    type PublicKey = P::PublicKey;
    fn get_pk(&self) -> &P::PublicKey {
        self.game.get_pk()
    }
}
impl<P, S> LeftOrRight for FromHybridToPubEnc<P, S>
where
    P: PubEnc,
    S: SymEnc,
    P: PubEnc<Plaintext = S::Key>,
    Standard: Distribution<S::Key>,
{
    type Plaintext = S::Plaintext;
    type Ciphertext = (P::Ciphertext, S::Ciphertext);
    fn left_or_right(
        &self,
        m_left: &S::Plaintext,
        m_right: &S::Plaintext,
    ) -> (P::Ciphertext, S::Ciphertext) {
        let mut rng = thread_rng();
        let tk_left = rng.gen();
        let tk_right = rng.gen();
        // Simulation: If `self.left`, then transition from `G1` to `G2`;
        // otherwise, transition from `G3` to `G4`. If `game.left`, then
        // simulate the former; otherwise simulate the latter.
        let c_pub = self.game.left_or_right(&tk_left, &tk_right);
        let c_sym = if self.sim_right {
            self.sym_enc.encrypt(&tk_left, m_right)
    } else {
            self.sym_enc.encrypt(&tk_left, m_left)
        };
        (c_pub, c_sym)
    }
}
\end{lstlisting}
%
The reduction is initialized with an instance of the $\code{PubCpa}$ game for
$\code{pub_enc}$, denoted by $\code{game}$. It simulates the game for
$\code{hybrid}$, except that instead of encrypting the left or right key
directly, it invokes $\code{game.left_or_right()}$ and uses the response to
produce the ciphertext. (See line 50.) This way the reduction simulaters
$\code{G1}$ when $\code{game.left == false}$ and $\code{G2}$ otherwise.

Let $R_{\code{12}}^A(\code{game}) =
A\code{.play(FromHybridToPubEnc::init(game), sym_enc, false))}$.
%
We wave the following.
%
%\cp{Here again is a step that we need to prove formally. Like the previous
%claim, our goal is to prove that two pieces of code are equivalent.}

\begin{claim}[Reduction]\label{claim/proof/hybrid/12}
  For all~$A$,
  \[
    \Prob{A\code{.play(G1::init(hybrid))}} =
    \Prob{R_{\code{12}}^A(\code{PubEnc::init(pub_enc, false)})}
  \]
  and
  \[
    \Prob{A\code{.play(G2::init(hybrid))}} =
    \Prob{R_{\code{12}}^A(\code{PubEnc::init(pub_enc, true)})} \,.
  \]
\end{claim}
%
Then by definition,
%
\[
  \Prob{A\code{.play(G1::init(hybrid))}} -
  \Prob{A\code{.play(G2::init(hybrid))}} \leq
  \Adv{cpa}_\code{pub_enc}(R_{\code{12}}^A) \,.
\]

Next, we obtain $\code{G3}$ from $\code{G2}$ by patching the
$\code{LeftOrRight}$ oracle once more:
%
\begin{lstlisting}[style=patch]
     fn left_or_right(
         &self,
-        m_left: &S::Plaintext,
-        _m_right: &S::Plaintext,
+        _m_left: &S::Plaintext,
+        m_right: &S::Plaintext,
     ) -> (P::Ciphertext, S::Ciphertext) {
         let mut rng = thread_rng();
         let tk_left = rng.gen();
-        let tk_right = rng.gen(); // new temporary key
+        let tk_right = rng.gen();
         let c_pub = self.enc.pub_enc.encrypt(&self.pk, &tk_right);
-        let c_sym = self.enc.sym_enc.encrypt(&tk_left, m_left);
+        let c_sym = self.enc.sym_enc.encrypt(&tk_left, m_right);
         (c_pub, c_sym)
     }
\end{lstlisting}
%
Now instead of encrypting the left plaintext, in the new game we encrypt the
right plaintext. Because $\code{c_pub}$ is independent of $\code{c_sym}$, we
can reduce the adversary's advantage in distinguishing $\code{G2}$ from
$\code{G2}$ to its advantage in breaking $\code{sym_enc}$.
%
Consider the following reduction:
%
\begin{lstlisting}
pub struct FromHybridToSymEnc<P, S>
where
    P: PubEnc,
    S: SymEnc,
{
    pub_enc: P,
    pk: P::PublicKey,
    game: Cpa<S>, // Instance of the game for `sym_enc`
}
impl<P, S> FromHybridToSymEnc<P, S>
where
    P: PubEnc,
    S: SymEnc,
{
    pub fn init(game: Cpa<S>, pub_enc: P) -> Self {
        let (pk, _sk) = pub_enc.key_gen();
        Self { pub_enc, pk, game }
    }
}
impl<P, S> GetPublicKey for FromHybridToSymEnc<P, S>
where
    P: PubEnc,
    S: SymEnc,
{
    type PublicKey = P::PublicKey;
    fn get_pk(&self) -> &P::PublicKey {
        &self.pk
    }
}
impl<P, S> LeftOrRight for FromHybridToSymEnc<P, S>
where
    P: PubEnc,
    S: SymEnc,
    P: PubEnc<Plaintext = S::Key>,
    Standard: Distribution<S::Key>,
{
    type Plaintext = S::Plaintext;
    type Ciphertext = (P::Ciphertext, S::Ciphertext);
    fn left_or_right(
        &self,
        m_left: &S::Plaintext,
        m_right: &S::Plaintext,
    ) -> (P::Ciphertext, S::Ciphertext) {
        let tk_right = thread_rng().gen();
        let c_pub = self.pub_enc.encrypt(&self.pk, &tk_right);
        // Simulation: if `game.left`, then this simulates `G2`; otherwise, this
        // simulates `G3`.
        let c_sym = self.game.left_or_right(m_left, m_right);
        (c_pub, c_sym)
    }
}
\end{lstlisting}
%
This time the reduction gets an instance of the
\hyperref[fig/symenc/security]{$\code{Cpa}$} game for $\code{sym_enc}$. It
simulates the $\code{PubCpa}$ game for $\code{hybrid}$ by generating a dummy
temporary key $\code{tk_right}$ and encrypting it to the public key generated at
initialization. The symmetric ciphertext is produced by invoking
$\code{game.left_or_right(m_left, m_right)}$; by construction, the encryption
key is independent of $\code{c_pub}$, as it is in both games.
%
Similar to the previous step, when $\code{game.right == false}$, the reduction
simulates $\code{G2}$ and simulates $\code{G3}$ otherwise.

Let $R_{\code{23}}^A(\code{game}) =
A\code{.play(FromHybridToSymEnc::init(game), pub_enc))}$.
%
We have the following.
%
%\cp{Here again is a step that we need to prove formally.}

\begin{claim}[Reduction]
  For all~$A$,
  \[
    \Prob{A\code{.play(G2::init(hybrid))}} =
    \Prob{R_{\code{23}}^A(\code{Cpa::init(sym_enc, false)})}
  \]
  and
  \[
    \Prob{A\code{.play(G3::init(hybrid))}} =
    \Prob{R_{\code{23}}^A(\code{Cpa::init(sym_enc, true)})} \,.
  \]
\end{claim}
%
Then by definition,
%
\[
  \Prob{A\code{.play(G2::init(hybrid))}} -
  \Prob{A\code{.play(G3::init(hybrid))}} \leq
  \Adv{cpa}_\code{sym_enc}(R_{\code{23}}^A) \,.
\]

Finally, we obtain $\code{G4}$ from $\code{G3}$ by applying the following patch:
%
\begin{lstlisting}[style=patch]
    fn left_or_right(
         &self,
         _m_left: &S::Plaintext,
         m_right: &S::Plaintext,
     ) -> (P::Ciphertext, S::Ciphertext) {
         let mut rng = thread_rng();
-        let tk_left = rng.gen();
+        let _tk_left = rng.gen(); // no longer used
         let tk_right = rng.gen();
         let c_pub = self.enc.pub_enc.encrypt(&self.pk, &tk_right);
-        let c_sym = self.enc.sym_enc.encrypt(&tk_left, m_right);
+        let c_sym = self.enc.sym_enc.encrypt(&tk_right, m_right); // use right temporary key
         (c_pub, c_sym)
     }
\end{lstlisting}
%
In the new game, we encrypt the right temporary key instead of the left. Here
again we bound the attacker's distinguishing advantage by reducing to CPA
security of $\code{pub_enc}$.
%
We can reuse $\code{FromHybridToPubEnc}$, except we fix $\code{sim_right ==
true}$.
%
Let $R_{\code{34}}^A(\code{game}) =
A\code{.play(FromHybridToPubEnc::init(game), sym_enc, true))}$.
%
We Have the following.

\begin{claim}[Reduction]
  For all~$A$,
  \[
    \Prob{A\code{.play(G3::init(hybrid))}} =
    \Prob{R_{\code{34}}^A(\code{PubEnc::init(pub_enc, false)})}
  \]
  and
  \[
    \Prob{A\code{.play(G4::init(hybrid))}} =
    \Prob{R_{\code{34}}^A(\code{PubEnc::init(pub_enc, true)})} \,.
  \]
\end{claim}
%
%\cp{The proof is pretty much the same as for Claim~\ref{claim/proof/hybrid/12};
%the only difference is that the setting of the $\code{right}$ flag. In fact, we
%may be able to prove both claims at once.}
%
Then by definition,
%
\[
  \Prob{A\code{.play(G3::init(hybrid))}} -
  \Prob{A\code{.play(G4::init(hybrid))}} \leq
  \Adv{cpa}_\code{pub_enc}(R_{\code{34}}^A) \,.
\]

Finally, game $\code{G4}$ is equivalent to $\code{PubCpa}$ with $\code{right ==
true}$.
%
\begin{claim}[Equivalent-rewrite]
  For all $A$,
  \[
     \Prob{A\code{.play(G1::init(hybrid))}} =
     \Prob{A\code{.play(PubCpa::init(hybrid, true))}} \,.
  \]
\end{claim}
%
%\cp{To be proven.}

So far we have that
%
\[
  \Adv{cpa}_\code{hybrid}(A) \leq
    \Adv{cpa}_\code{pub_cpa}(R_{\code{12}}^A) +
    \Adv{cpa}_\code{sym_cpa}(R_{\code{23}}^A) +
    \Adv{cpa}_\code{pub_cpa}(R_{\code{34}}^A) \,.
\]
%
Let $B = R_{\code{23}}^A$.
%
Define $C$ as follows. Flip a coin: if the outcome is $\code{false}$, then run
$R_{\code{12}}^A$; if the outcome is $\code{true}$, then run $R_{\code{34}}^A$.
%
Applying the law of total probability, we obtain
%
% adv(C) = pr(C|left) - pr(C|right)
%        = 1/2 pr(R12|left) + 1/2 pr(R34|left) - (1/2 pr(R12|right) + 1/2 pr(R34|right))
%        = 1/2 (pr(R12|left) - pr(R12|right)) + 1/2 (pr(R34|left) - pr(R34|right))
%        = 1/2 adv(R12) + 1/2 adv(R34)
%
\[
  \Adv{cpa}_\code{pub_cpa}(R_{\code{12}}^A) +
  \Adv{cpa}_\code{pub_cpa}(R_{\code{34}}^A)
  = 2\cdot \Adv{cpa}_\code{pub_cpa}(C) \,.
\]
%
And that's the way it is!

\end{proof}

\section{Example: The PRP/PRF Switching Lemma}

In this section we fix the PRP/PRF switching lemma~\cite{br06}, which states
that a good pseudorandom permutation (PRP) is a good pseudorandom function
(PRF) up to birthday attacks.
%
The proof demonstrates the \emph{equivalent-until-bad} transition.

\cp{TODO.}

\bibliographystyle{plain} % We choose the "plain" reference style
\bibliography{references}

\appendix

\section{Deferred Definitions}

\subsection{Symmetric Encryption}
\label{sec/symenc}

A symmetric encryption scheme implements the $\code{SymEnc}$ trait listed in
Figure~\ref{fig/symenc/syntax}.
%
Let $\code{enc: E}$ implement $\code{SymEnc}$.
%
We say $\code{enc}$ is \emph{correct} if for all $\code{m} \in
\code{E::Plaintext}$ it holds that
%
\[
  \Prob{
    \code{let k = e.key_gen();
    enc.decrypt(k, enc.encrypt(k, m)) == Some(m)}
  } = 1 \,.
\]
%
Security under chosen plaintext attack (CPA) is defined by the $\code{Cpa}$
game in Figure~\ref{fig/symenc/security}.

\begin{definition}[{\cite[Definition 7.1]{joy}}]
  Let $A$ implement \hyperref[sec/traits]{$\code{Distinguisher}$}.
  %
  Let~$\code{enc}$ implement \hyperref[fig/symenc/syntax]{$\code{SymEnc}$}.
  %
  Define the advantage of~$A$ in breaking the security of $\code{enc}$ under CPA
  as
  %
  \[
    \Adv{cpa}_{\code{enc}}(A) =
      \Prob{A\code{.play(Cpa::init(enc, false))}} -
      \Prob{A\code{.play(Cpa::init(enc, true))}} \,.
  \]
  %
  Informally, we say $\code{enc}$ is secure under CPA if every efficient adversary
  has small advantage.
\end{definition}

% ../src/joy/hybrid/mod.rs
\begin{figure}
\begin{lstlisting}
pub trait SymEnc {
    type Key;
    type Plaintext;
    type Ciphertext;
    fn key_gen(&self) -> Self::Key;
    fn encrypt(&self, k: &Self::Key, m: &Self::Plaintext) -> Self::Ciphertext;
    fn decrypt(&self, k: &Self::Key, c: &Self::Ciphertext) -> Option<Self::Plaintext>;
}
\end{lstlisting}
\begin{lstlisting}
pub struct Cpa<E: SymEnc> {
    e: E,
    k: E::Key,
    right: bool,
}
impl<E: SymEnc> Cpa<E> {
    pub fn init(e: E, right: bool) -> Self {
        let k = e.key_gen();
        Self { e, k, right }
    }
}
impl<E: SymEnc> LeftOrRight for Cpa<E> {
    type Plaintext = E::Plaintext;
    type Ciphertext = E::Ciphertext;
    fn left_or_right(
        &self,
        m_left: &E::Plaintext,
        m_right: &E::Plaintext,
    ) -> E::Ciphertext {
        if self.right {
            self.e.encrypt(&self.k, m_right)
        } else {
            self.e.encrypt(&self.k, m_left)
        }
    }
}
\end{lstlisting}
  \caption{Top: syntax of symmetric encryption. Bottom: game $\code{Cpa}$
  for defining security of symmetric encryption under chosen plaintext attack
  (CPA) of symmetric encryption.}
  \label{fig/symenc/syntax}
  \label{fig/symenc/security}
\end{figure}

\subsection{Traits}
\label{sec/traits}.

\begin{lstlisting}
/// Left-or-right encryption oracle.
pub trait LeftOrRight {
    type Plaintext;
    type Ciphertext;
    fn left_or_right(
        &self,
        m_left: &Self::Plaintext,
        m_right: &Self::Plaintext,
    ) -> Self::Ciphertext;
}

/// Oracle for obtaining the public key in a public-key cryptosystem.
pub trait GetPublicKey {
    type PublicKey;
    fn get_pk(&self) -> &Self::PublicKey;
}

/// A generic distinguishing attacker.
///
/// The attacker gets as input a "game" and outputs a bit. The output is used
/// to define a notion of advantage.
pub trait Distinguisher<G> {
    fn play(&self, game: G) -> Result<bool, Error>;
}
\end{lstlisting}

\end{document}
