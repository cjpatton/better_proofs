The proof is by a game-playing argument.
%
We begin with game $\code{G1}$ listed below:

\begin{lstlisting}
pub struct G1<P: PubEnc, S> {
    enc: Hybrid<P, S>,
    pk: P::PublicKey,
}
impl<P: PubEnc, S> G1<P, S> {
    pub fn init(enc: Hybrid<P, S>) -> Self {
        let (pk, _sk) = enc.pub_enc.key_gen();
        Self { enc, pk }
    }
}
impl<P: PubEnc, S> GetPublicKey for G1<P, S> {
    type PublicKey = P::PublicKey;
    fn get_pk(&self) -> &P::PublicKey {
        &self.pk
    }
}
impl<P, S> LeftOrRight for G1<P, S>
where
    S: SymEnc,
    P: PubEnc<Plaintext = S::Key>,
    Standard: Distribution<S::Key>,
{
    type Plaintext = S::Plaintext;
    type Ciphertext = (P::Ciphertext, S::Ciphertext);
    fn left_or_right(
        &self,
        m_left: &S::Plaintext,
        _m_right: &S::Plaintext,
    ) -> (P::Ciphertext, S::Ciphertext) {
        let mut rng = thread_rng();
        let tk_left = rng.gen();
        let _tk_right = rng.gen(); // new temporary key (unused)
        let c_pub = self.enc.pub_enc.encrypt(&self.pk, &tk_left);
        let c_sym = self.enc.sym_enc.encrypt(&tk_left, m_left);
        (c_pub, c_sym)
    }
}
\end{lstlisting}
%
This game was obtained from \hyperref[fig/pubenc/security]{PubEnc} by fixing
$\code{right == false}$, unrolling the call to $\code{self.enc.encrypt()}$ in
the $\code{LeftOrRight}$ oracle, and adding code for generating a new temporary
key, $\code{_tk_right}$. Let us call this new key (unused for the moment) the
``right key'' and the existing key the ``left key''.
%
\begin{claim}[Equivalent-rewrite]
  For all $A$,
  \[
     \Prob{A\code{.play(PubCpa::init(hybrid, false))}} =
     \Prob{A\code{.play(G1::init(hybrid))}} \,.
  \]
\end{claim}
%
\cp{The main goal for this document is to identify the best methodology for
establishing claims like this. The goal here seems simple, at least in
principle: prove that the programs in this equality are equivalent, and thus
the output being $\code{true}$ is equiprobable in both experiments.}

Next, game $\code{G2}$ is obtained by applying the following patch to the
$\code{LeftOrRight}$ oracle. By this we mean removing the lines beginning with
``$\code{-}$'' and and adding the lines beginning with ``$\code{+}$'':
%
\begin{lstlisting}[style=patch]
    fn left_or_right(
        &self,
        m_left: &S::Plaintext,
-       _m_right: &S::Plaintext,
+       m_right: &S::Plaintext,
    ) -> (P::Ciphertext, S::Ciphertext) {
         let mut rng = thread_rng();
         let tk_left = rng.gen();
-        let _tk_right = rng.gen(); // new temporary key (unused)
-        let c_pub = self.enc.pub_enc.encrypt(&self.pk, &tk_left);
+        let tk_right = rng.gen(); // new temporary key
+        let c_pub = self.enc.pub_enc.encrypt(&self.pk, &tk_right);
         let c_sym = self.enc.sym_enc.encrypt(&tk_left, m_left);
         (c_pub, c_sym)
     }
\end{lstlisting}
%
The patched oracle encrypts the right key rather than the left. The left key is
still used to produce $\code{s_sym}$, so $\code{c_sym}$ is now independent of
$\code{c_pub}$.

We can reduce the attacker's ability to distinguish between $\code{G1}$ and
$\code{G2}$ to the CPA security of $\code{pub_cpa}$.
%
Consider the following reduction, which transforms a $\code{hybrid}$ attacker
into a $\code{pub_enc}$ attacker. For the moment, fix $\code{right == false}$;
we will use $\code{right == true}$ in a later step.
%
\begin{lstlisting}
pub struct FromHybridToPubEnc<P, S>
where
    P: PubEnc,
{
    sym_enc: S,
    game: PubCpa<P>, // Instance of the game for `pub_enc`.
    right: bool,
}
impl<P, S> FromHybridToPubEnc<P, S>
where
    P: PubEnc,
{
    pub fn init(game: PubCpa<P>, sym_enc: S, right: bool) -> Self {
        Self {
            game,
            sym_enc,
            right,
        }
    }
}
impl<P, S> GetPublicKey for FromHybridToPubEnc<P, S>
where
    P: PubEnc,
{
    type PublicKey = P::PublicKey;
    fn get_pk(&self) -> &P::PublicKey {
        self.game.get_pk()
    }
}
impl<P, S> LeftOrRight for FromHybridToPubEnc<P, S>
where
    P: PubEnc,
    S: SymEnc,
    P: PubEnc<Plaintext = S::Key>,
    Standard: Distribution<S::Key>,
{
    type Plaintext = S::Plaintext;
    type Ciphertext = (P::Ciphertext, S::Ciphertext);
    fn left_or_right(
        &self,
        m_left: &S::Plaintext,
        m_right: &S::Plaintext,
    ) -> (P::Ciphertext, S::Ciphertext) {
        let mut rng = thread_rng();
        let tk_left = rng.gen();
        let tk_right = rng.gen();
        // Simulation: If `self.left`, then transition from `G1` to `G2`;
        // otherwise, transition from `G3` to `G4`. If `game.left`, then
        // simulate the former; otherwise simulate the latter.
        let c_pub = self.game.left_or_right(&tk_left, &tk_right);
        let c_sym = if self.right {
            self.sym_enc.encrypt(&tk_left, m_right)
    } else {
            self.sym_enc.encrypt(&tk_left, m_left)
        };
        (c_pub, c_sym)
    }
}
\end{lstlisting}
%
The reduction is initialized with an instance of the $\code{PubCpa}$ game for
$\code{pub_enc}$, denoted by $\code{game}$. It simulates the game for
$\code{hybrid}$, except that instead of encrypting the left or right key
directly, it invokes $\code{game.left_or_right()}$ and uses the response to
produce the ciphertext. (See line 50.) This way the reduction simulaters
$\code{G1}$ when $\code{game.left == false}$ and $\code{G2}$ otherwise.

Let $R_{\code{12}}^A(\code{game}) =
A\code{.play(FromHybridToPubEnc::init(game), sym_enc, false))}$.
%
We Have the following.

\begin{claim}[Equivalent-by-reduction]\label{claim/proof/hybrid/12}
  For all~$A$,
  \[
    \Prob{A\code{.play(G1::init(hybrid))}} =
    \Prob{R_{\code{12}}^A(\code{PubEnc::init(pub_enc, false)})}
  \]
  and
  \[
    \Prob{A\code{.play(G2::init(hybrid))}} =
    \Prob{R_{\code{12}}^A(\code{PubEnc::init(pub_enc, true)})} \,.
  \]
\end{claim}
%
\cp{Here again is a step that we need to prove formally. Like the previous
claim, our goal is to prove that two pieces of code are equivalent.}
%
Then by definition,
%
\[
  \Prob{A\code{.play(G1::init(hybrid))}} -
  \Prob{A\code{.play(G2::init(hybrid))}} \leq
  \Adv{cpa}_\code{pub_enc}(R_{\code{12}}^A) \,.
\]

Next, we obtain $\code{G3}$ from $\code{G2}$ by patching the
$\code{LeftOrRight}$ oracle once more:
%
\begin{lstlisting}[style=patch]
     fn left_or_right(
         &self,
-        m_left: &S::Plaintext,
-        _m_right: &S::Plaintext,
+        _m_left: &S::Plaintext,
+        m_right: &S::Plaintext,
     ) -> (P::Ciphertext, S::Ciphertext) {
         let mut rng = thread_rng();
         let tk_left = rng.gen();
-        let tk_right = rng.gen(); // new temporary key
+        let tk_right = rng.gen();
         let c_pub = self.enc.pub_enc.encrypt(&self.pk, &tk_right);
-        let c_sym = self.enc.sym_enc.encrypt(&tk_left, m_left);
+        let c_sym = self.enc.sym_enc.encrypt(&tk_left, m_right);
         (c_pub, c_sym)
     }
\end{lstlisting}
%
Now instead of encrypting the left plaintext, in the new game we encrypt the
right plaintext. Because $\code{c_pub}$ is independent of $\code{c_sym}$, we
can reduce the adversary's advantage in distinguishing $\code{G2}$ from
$\code{G2}$ to its advantage in breaking $\code{pub_enc}$.
%
Consider the following reduction:
%
\begin{lstlisting}
pub struct FromHybridToSymEnc<P, S>
where
    P: PubEnc,
    S: SymEnc,
{
    pub_enc: P,
    pk: P::PublicKey,
    game: Cpa<S>, // Instance of the game for `sym_enc`
}
impl<P, S> FromHybridToSymEnc<P, S>
where
    P: PubEnc,
    S: SymEnc,
{
    pub fn init(game: Cpa<S>, pub_enc: P) -> Self {
        let (pk, _sk) = pub_enc.key_gen();
        Self { pub_enc, pk, game }
    }
}
impl<P, S> GetPublicKey for FromHybridToSymEnc<P, S>
where
    P: PubEnc,
    S: SymEnc,
{
    type PublicKey = P::PublicKey;
    fn get_pk(&self) -> &P::PublicKey {
        &self.pk
    }
}
impl<P, S> LeftOrRight for FromHybridToSymEnc<P, S>
where
    P: PubEnc,
    S: SymEnc,
    P: PubEnc<Plaintext = S::Key>,
    Standard: Distribution<S::Key>,
{
    type Plaintext = S::Plaintext;
    type Ciphertext = (P::Ciphertext, S::Ciphertext);
    fn left_or_right(
        &self,
        m_left: &S::Plaintext,
        m_right: &S::Plaintext,
    ) -> (P::Ciphertext, S::Ciphertext) {
        let tk_right = thread_rng().gen();
        let c_pub = self.pub_enc.encrypt(&self.pk, &tk_right);
        // Simulation: if `game.left`, then this simulates `G2`; otherwise, this
        // simulates `G3`.
        let c_sym = self.game.left_or_right(m_left, m_right);
        (c_pub, c_sym)
    }
}
\end{lstlisting}
%
This time the reduction gets an instance of the
\hyperref[fig/symenc/security]{$\code{Cpa}$} game for $\code{sym_enc}$. It
simulates the $\code{PubCpa}$ game for $\code{hybrid}$ by generating a dummy
temporary key $\code{tk_right}$ and encrypting it to the public key generated at
initialization. The symmetric ciphertext is produced by invoking
$\code{game.left_or_right(m_left, m_right)}$; by construction, the encryption
key is independent of $\code{c_pub}$, as it is in both games.
%
Similar to the previous step, when $\code{game.right == false}$, the reduction
simulates $\code{G2}$ and simulates $\code{G3}$ otherwise.

Let $R_{\code{23}}^A(\code{game}) =
A\code{.play(FromHybridToSymEnc::init(game), pub_enc))}$.
%
We Have the following.

\begin{claim}[Equivalent-by-reduction]
  For all~$A$,
  \[
    \Prob{A\code{.play(G2::init(hybrid))}} =
    \Prob{R_{\code{23}}^A(\code{Cpa::init(sym_enc, false)})}
  \]
  and
  \[
    \Prob{A\code{.play(G3::init(hybrid))}} =
    \Prob{R_{\code{23}}^A(\code{Cpa::init(sym_enc, true)})} \,.
  \]
\end{claim}
%
\cp{Here again is a step that we need to prove formally.}
%
Then by definition,
%
\[
  \Prob{A\code{.play(G2::init(hybrid))}} -
  \Prob{A\code{.play(G3::init(hybrid))}} \leq
  \Adv{cpa}_\code{sym_enc}(R_{\code{23}}^A) \,.
\]

Finally, we obtain $\code{G4}$ from $\code{G3}$ by applying the following patch:
%
\begin{lstlisting}[style=patch]
    fn left_or_right(
         &self,
         _m_left: &S::Plaintext,
         m_right: &S::Plaintext,
     ) -> (P::Ciphertext, S::Ciphertext) {
         let mut rng = thread_rng();
-        let tk_left = rng.gen();
+        let _tk_left = rng.gen(); // no longer used
         let tk_right = rng.gen();
         let c_pub = self.enc.pub_enc.encrypt(&self.pk, &tk_right);
-        let c_sym = self.enc.sym_enc.encrypt(&tk_left, m_right);
+        let c_sym = self.enc.sym_enc.encrypt(&tk_right, m_right); // use right temporary key
         (c_pub, c_sym)
     }
\end{lstlisting}
%
In the new game, we encrypt the right temporary key instead of the left. Were
again we bound the attacker's distinguishing advantage by reducing to CPA
security of $\code{pub_enc}$.
%
We can reuse $\code{FromHybridToPubEnc}$, except we fix $\code{right == true}$.
%
Let $R_{\code{34}}^A(\code{game}) =
A\code{.play(FromHybridToPubEnc::init(game), sym_enc, true))}$.
%
We Have the following.

\begin{claim}[Equivalent-by-reduction]
  For all~$A$,
  \[
    \Prob{A\code{.play(G3::init(hybrid))}} =
    \Prob{R_{\code{34}}^A(\code{PubEnc::init(pub_enc, false)})}
  \]
  and
  \[
    \Prob{A\code{.play(G4::init(hybrid))}} =
    \Prob{R_{\code{34}}^A(\code{PubEnc::init(pub_enc, true)})} \,.
  \]
\end{claim}
%
\cp{The proof is pretty much the same as for Claim~\ref{claim/proof/hybrid/12};
the only difference is that the setting of the $\code{right}$ flag. In fact, we
may be able to prove both claims at once.}
%
It follows by definition that
%
Then by definition,
%
\[
  \Prob{A\code{.play(G3::init(hybrid))}} -
  \Prob{A\code{.play(G4::init(hybrid))}} \leq
  \Adv{cpa}_\code{pub_enc}(R_{\code{34}}^A) \,.
\]

Finally, game $\code{G4}$ is equivalent to $\code{PubCpa}$ with $\code{right ==
true}$.
%
\begin{claim}[Equivalent-rewrite]
  For all $A$,
  \[
     \Prob{A\code{.play(G1::init(hybrid))}} =
     \Prob{A\code{.play(PubCpa::init(hybrid, true))}} \,.
  \]
\end{claim}
%
\cp{To be proven.}

So far we have that
%
\[
  \Adv{cpa}_\code{hybrid}(A) \leq
    \Adv{cpa}_\code{pub_cpa}(R_{\code{12}}^A) +
    \Adv{cpa}_\code{sym_cpa}(R_{\code{23}}^A) +
    \Adv{cpa}_\code{pub_cpa}(R_{\code{34}}^A) \,.
\]
%
Let $B = R_{\code{23}}^A$.
%
Define $C$ as follows. Flip a coin: if the outcome is $\code{false}$, then run
$R_{\code{12}}^A$; if the outcome is $\code{true}$, then run $R_{\code{34}}^A$.
%
Applying the law of total probability,
%
\[
  \Adv{cpa}_\code{pub_cpa}(R_{\code{12}}^A) +
  \Adv{cpa}_\code{pub_cpa}(R_{\code{34}}^A)
  = 2\cdot \Adv{cpa}_\code{pub_cpa}(C) \,.
\]
%
The claimed bound follows.
