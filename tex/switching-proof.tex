We first observe that
%
% adv(prp) = pr(real) - pr(func)
%          = pr(real) - pr(perm) + pr(func) - pr(func)
%          = pr(real) - pr(func) + pr(func) - pr(perm)
%          = adv(prf) + pr(func) - pr(perm)
\[
  \Adv{prp}_\code{f}(A) =
      \Adv{prf}_\code{f}(A) +
      \Prob{A\code{.play(RandFunc::default())}} -
      \Prob{A\code{.play(RandPerm::default())}} \,.
\]
%
We consider the difference $\Prob{A\code{.play(RandFunc::default())}} -
\Prob{A\code{.play(RandPerm::default())}}$ in the remainder.
%
We begin with the following game:
%
\begin{lstlisting}
#[derive(Default)]
pub struct G1<F: Func>
where
    F::Domain: Sized,
{
    table: HashMap<F::Domain, F::Range>,
    range: HashSet<F::Range>,
    bad: bool,
}
impl<F: Func> Eval<F> for G1<F>
where
    Standard: Distribution<F::Range>,
    F::Domain: Clone + std::hash::Hash + Eq,
    F::Range: Clone + std::hash::Hash + Eq,
{
    fn eval(&mut self, x: &F::Domain) -> F::Range {
        let mut rng = thread_rng();
        self.table
            .entry(x.clone())
            .or_insert(loop {
                let y = rng.gen();
                if self.range.contains(&y) {
                    self.bad = true;
                }
                self.range.insert(y.clone());
                break y;
            })
            .clone()
    }
}
\end{lstlisting}
%
This game was obtained from \hyperref[sec/func/ideal]{$\code{RandFunc}$} by
modifying the closure passed to $\code{.or_insert()}$. The new code samples the
range value as usual, but does some extra book-keeping that sets us up for the
next step. In particular, the sampled point, denoted $\code{y}$, is stored in a
set $\code{range}$: if ever $\code{y}$ is already in $\code{range}$, then the
game sets a flag called $\code{bad}$, but otherwise proceeds as usual.

\begin{claim}[Equivalent-rewrite]
  For all $A$,
  \[
     \Prob{A\code{.play(RandFunc::default())}} =
     \Prob{A\code{.play(G1::default())}} \,.
  \]
\end{claim}
%
%\cp{To be proven.}

Next, game $\code{G2}$ is obtained from $\code{G2}$ by applying the following patch:
%
\begin{lstlisting}[style=patch]
    fn eval(&mut self, x: &F::Domain) -> F::Range {
        let mut rng = thread_rng();
        self.table
            .entry(x.clone())
            .or_insert(loop {
                 let y = rng.gen();
                 if self.range.contains(&y) {
                     self.bad = true;
+                    continue; // reject and try again
                 }
                 self.range.insert(y.clone());
                 break y;
            })
            .clone()
    }
\end{lstlisting}
%
This modifies the code after $\code{bad}$ gets set. In the new game, if there
is a collision in the range, then~$\code{y}$ is rejected and we try again.
%
Let $\bad = A\code{.play_then_return(G2::default()).bad}$.
%
\begin{claim}[Equivalent-until-bad]
  For all $A$,
  \[
     \Prob{A\code{.play(G1::default())}} -
     \Prob{A\code{.play(G2::default())}} \leq \Prob{\bad} \,.
  \]
\end{claim}
%
%\cp{To be proven.}

Finally, game $\code{G2}$ is equivalent to
\hyperref[sec/func/ideal]{$\code{RandPerm}$}. Namely, the latter is obtained
from the former by negating the condition and re-arranging the code.
%
\begin{claim}[Equivalent-rewrite]
  For all $A$,
  \[
     \Prob{A\code{.play(G2::default())}} =
     \Prob{A\code{.play(RandPerm::default())}} \,.
  \]
\end{claim}
%
%\cp{To be proven.}

To obtain the final bound, observe that the $\bad$ event occurs if, at any
point in the game's execution, it holds that $\code{y}$ is contained in
$\code{range}$.
%
This can occur on any \hyperref[sec/traits]{$\code{Eval}$} query, of which
there at at most~$q$.
%
Since each range point is sampled uniform randomly, we have that
%
\[
  \Prob{\bad} \leq
    \binom{q}{2} \cdot \frac{1}{|\code{F::Range}|} =
    \frac{q(q-1)}{2|\code{F::Range}|} \,.
\]
